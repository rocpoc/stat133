\documentclass[]{article}
\usepackage{lmodern}
\usepackage{amssymb,amsmath}
\usepackage{ifxetex,ifluatex}
\usepackage{fixltx2e} % provides \textsubscript
\ifnum 0\ifxetex 1\fi\ifluatex 1\fi=0 % if pdftex
  \usepackage[T1]{fontenc}
  \usepackage[utf8]{inputenc}
\else % if luatex or xelatex
  \ifxetex
    \usepackage{mathspec}
  \else
    \usepackage{fontspec}
  \fi
  \defaultfontfeatures{Ligatures=TeX,Scale=MatchLowercase}
\fi
% use upquote if available, for straight quotes in verbatim environments
\IfFileExists{upquote.sty}{\usepackage{upquote}}{}
% use microtype if available
\IfFileExists{microtype.sty}{%
\usepackage{microtype}
\UseMicrotypeSet[protrusion]{basicmath} % disable protrusion for tt fonts
}{}
\usepackage[margin=1in]{geometry}
\usepackage{hyperref}
\hypersetup{unicode=true,
            pdftitle={Warm Up 01},
            pdfauthor={Rocky Lubbers},
            pdfborder={0 0 0},
            breaklinks=true}
\urlstyle{same}  % don't use monospace font for urls
\usepackage{longtable,booktabs}
\usepackage{graphicx,grffile}
\makeatletter
\def\maxwidth{\ifdim\Gin@nat@width>\linewidth\linewidth\else\Gin@nat@width\fi}
\def\maxheight{\ifdim\Gin@nat@height>\textheight\textheight\else\Gin@nat@height\fi}
\makeatother
% Scale images if necessary, so that they will not overflow the page
% margins by default, and it is still possible to overwrite the defaults
% using explicit options in \includegraphics[width, height, ...]{}
\setkeys{Gin}{width=\maxwidth,height=\maxheight,keepaspectratio}
\IfFileExists{parskip.sty}{%
\usepackage{parskip}
}{% else
\setlength{\parindent}{0pt}
\setlength{\parskip}{6pt plus 2pt minus 1pt}
}
\setlength{\emergencystretch}{3em}  % prevent overfull lines
\providecommand{\tightlist}{%
  \setlength{\itemsep}{0pt}\setlength{\parskip}{0pt}}
\setcounter{secnumdepth}{0}
% Redefines (sub)paragraphs to behave more like sections
\ifx\paragraph\undefined\else
\let\oldparagraph\paragraph
\renewcommand{\paragraph}[1]{\oldparagraph{#1}\mbox{}}
\fi
\ifx\subparagraph\undefined\else
\let\oldsubparagraph\subparagraph
\renewcommand{\subparagraph}[1]{\oldsubparagraph{#1}\mbox{}}
\fi

%%% Use protect on footnotes to avoid problems with footnotes in titles
\let\rmarkdownfootnote\footnote%
\def\footnote{\protect\rmarkdownfootnote}

%%% Change title format to be more compact
\usepackage{titling}

% Create subtitle command for use in maketitle
\newcommand{\subtitle}[1]{
  \posttitle{
    \begin{center}\large#1\end{center}
    }
}

\setlength{\droptitle}{-2em}

  \title{Warm Up 01}
    \pretitle{\vspace{\droptitle}\centering\huge}
  \posttitle{\par}
    \author{Rocky Lubbers}
    \preauthor{\centering\large\emph}
  \postauthor{\par}
    \date{}
    \predate{}\postdate{}
  

\begin{document}
\maketitle

\section{Jon Snow}\label{jon-snow}

\begin{figure}
\centering
\includegraphics{https://upload.wikimedia.org/wikipedia/en/f/f0/Jon_Snow-Kit_Harington.jpg}
\caption{}
\end{figure}

\begin{quote}
\emph{``When enough people make false promises,\emph{ }words stop
meaning anything'' -- Jon Snow}
\end{quote}

\begin{longtable}[]{@{}ll@{}}
\toprule
Jon Snow & Facts\tabularnewline
\midrule
\endhead
Status & Alive\tabularnewline
Date of Birth & 281 AC\tabularnewline
Origin & Winterfell\tabularnewline
Allegiance & House Stark\tabularnewline
Predecessor & N/A\tabularnewline
Successor & N/A\tabularnewline
Culture & Northmen\tabularnewline
Religion & Old Gods of the Forest\tabularnewline
Father & Rhaegar Targaryen\tabularnewline
Mother & Lyanna Stark\tabularnewline
\bottomrule
\end{longtable}

\begin{center}\rule{0.5\linewidth}{\linethickness}\end{center}

\subsection{Corn Fritter Recipe}\label{corn-fritter-recipe}

\begin{figure}
\centering
\includegraphics{https://www.eatwell101.com/wp-content/uploads/2018/05/Cheesy-Corn-Fritters-3.jpg}
\caption{}
\end{figure}

\begin{center}\rule{0.5\linewidth}{\linethickness}\end{center}

Ingredients

\begin{itemize}
\tightlist
\item
  6 ears of corn (about 3 cups corn)
\item
  4 scallions, both white and greens finely chopped
\item
  1/2 cup chopped herbs of your choice (chives are the go-to for me)
\item
  About 1 cup (6 ounces) grated sharp cheddar (other cheese is fine too)
\item
  3/4 teaspoon kosher salt, plus more to taste
\item
  Freshly ground black pepper
\item
  4 large eggs
\item
  1 cup all-purpose flour, plus 2 more tablespoons if needed
\item
  Olive or a neutral oil for frying (you can use Safflower or Avocado
  oil!)
\end{itemize}

\begin{center}\rule{0.5\linewidth}{\linethickness}\end{center}

Special Kitchen Tools

\begin{itemize}
\tightlist
\item
  Sharp Knife
\item
  Medium-Large Pan
\item
  Paper Towels (for dabbing oil)
\end{itemize}

\begin{center}\rule{0.5\linewidth}{\linethickness}\end{center}

Steps

\begin{enumerate}
\def\labelenumi{\arabic{enumi}.}
\tightlist
\item
  Use a sharp knife to cut the kernels from the corn into the bowl, then
  run the back of your knife up and down the stalk to release as much
  ``milk'' as possible into the bowl. Try to fill around 3 cups of corn.
\item
  Add scallions, herbs, cheese, and many grinds of black pepper and stir
  to evenly combine.
\item
  Add the eggs and use a fork or spoon to stir until they're all broken
  up and evenly coat the corn mixture.
\item
  Add 1 cup of flour and stir to throughly coat.
\item
  Heat 2 to 3 tablespoons oil in a large frying pan over medium heat.
  Once hot and shimmering, add your first scoop of corn fritter batter
  and press it gently to flatten it. \textbf{Corn fritters cook quickly
  so keep an eye on them.} When the underside is a deep golden brown,
  flip and cook to the same color on the second side.
\item
  Drain on a paper towel, sprinkling on more salt. When it's cool enough
  to try, taste and adjust the seasonings of the remaining batter if
  needed.
\item
  \textbf{Enjoy!}
\end{enumerate}

\begin{center}\rule{0.5\linewidth}{\linethickness}\end{center}

\subsection{Quadratic Formula}\label{quadratic-formula}

A quadratic equation is any equation having the form:

\begin{center}\rule{0.5\linewidth}{\linethickness}\end{center}

\(ax^2 + bx + c = 0\)

\emph{where x represents a variable, and a, b, and c represent known
numbers such that a is not equal to 0}

\begin{center}\rule{0.5\linewidth}{\linethickness}\end{center}

The formula for the \emph{roots} of the quadratic equation goes as
follows:

\(x={\frac {-b\pm {\sqrt {b^{2}-4ac\ }}}{2a}}.\)

\begin{center}\rule{0.5\linewidth}{\linethickness}\end{center}

Furthermore, the derivation of the solution is:

\(ax^2 + bx + c = 0\)

\(4a^2x^2 + 4abx + 4ac = 0\)

\(4a^2x^2 + 4abx = - 4ac\)

\(4a^2x^2 + 4abx + b^2 = b^2 - 4ac\)

\((2ax + b)^2 = b^2 - 4ac\)

\(2ax + b = \pm \sqrt{b^2 - 4ac}\)

\(2ax = -b \pm \sqrt{b^2 - 4ac}\)

\(x= {\frac {-b\pm {\sqrt {b^{2}-4ac\ }}}{2a}}.\)

\begin{center}\rule{0.5\linewidth}{\linethickness}\end{center}

\subsubsection{Comments \& Reflections}\label{comments-reflections}

Since this was my first time using markdown, some of the intricate Latex
mathematical symbols started to become somewhat tedious and frustrating,
however most of the other writing was straightforward and easy for me.
It took me around an hour to complete this assignment start to finish by
myself. I did not need help on this, and I noticed most of my questions
I could look up documentation for which is comforting! I think the most
time consuming part for me however was the math section as well as
putting an image in for the Game of Thrones part!

\begin{center}\rule{0.5\linewidth}{\linethickness}\end{center}


\end{document}
